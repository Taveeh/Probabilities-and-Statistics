\documentclass{article}
\usepackage[utf8]{inputenc}
\usepackage[english]{babel}
\usepackage{amsthm}
\usepackage{amsmath}
\usepackage{amssymb}


\title{Seminar 5}
\author{Octavian Custura}
\date{December 2020}

\begin{document}

\maketitle

\textbf{7.} An internet service provider has two connection lines for its customers. Eighty percent of customers are
connected through Line I, the rest through Line II. Line I has a $Gamma(3, \frac{1}{2})$ connection time (in minutes),
while Line II has a $U(20, 50)$ connection time (in seconds). Find the probability that it takes a randomly selected
customer more than 30 seconds to connect to the internet.

\begin{proof}

Firstly, we want to compute the probability that it takes a randomly selected customer from Line I more than 30 seconds to connect to the internet. We know that the connection time is $Gamma(3, \frac{1}{2})$, so we can compute
\begin{equation*}
    \text{pdf } f(x) = 4{x}^{2}e^{-2x} \\
\end{equation*}
\begin{equation*}
        F(\frac{1}{2}) = P(X \leq \frac{1}{2}) = \int_{-\infty}^{\frac{1}{2}} f(x)\,dx = \int_{0}^{\frac{1}{2}} f(x)\,dx
\end{equation*}
since one cannot wait $< 0$ minutes.
\begin{equation*}
    \int 4{x}^{2}e^{-2x} \,dx = -\left(2x^2+2x+1\right)\mathrm{e}^{-2x} + C
\end{equation*}
\begin{equation*}
    \text{Thus, } F(\frac{1}{2}) \approx 0.08030
\end{equation*}
\begin{equation*}
    \begin{split}
    P(T > \frac{1}{2} \text{min}) &= 1 - P(T \leq \frac{1}{2} \text{min}) \\
    &= 1 - F(\frac{1}{2}) \\
    &= 0.91970 = P1
    \end{split}
\end{equation*}
\textbf{or}
\begin{equation}
\begin{split}
    P(T > \frac{1}{2} \text{min}) &= 1 - P(T \leq \frac{1}{2} \text{min}) \\
    &= 1 - gamcdf(\frac{1}{2}, 3, \frac{1}{2}) \textit{ (in matlab)} \\
    &= 0.91970 = P1
\end{split}
\end{equation}

Secondly, we want to compute the probability that it takes a randomly selected customer from Line II more than 30 seconds to connect to the internet. We know that the connection time is $U(20, 50)$, so we can compute
\begin{equation*}
    \text{pdf } f(x) = \frac{1}{30} \text{, } x \in [20, 50] 
\end{equation*}
\begin{equation*}
        F(30) = P(X \leq 30) = \int_{-\infty}^{30} f(x)\,dx = \int_{20}^{30} f(x)\,dx
\end{equation*}
since $f$ is defined on $[20, 50]$
\begin{equation*}
    \int \frac{1}{30} \,dx = \frac{x}{30} + C
\end{equation*}
\begin{equation*}
    \text{Thus, } F(30) = \frac{30}{30} - \frac{20}{30} = \frac{1}{3} = 0.3333 
\end{equation*}
\begin{equation*}
\begin{split}
    P(T > 30 \text{sec}) &= 1 - P(T \leq 30 \text{sec}) \\
    &= 1 - F(30)\\
    &= 0.66667 = P2
\end{split}
\end{equation*}
\textbf{or}
\begin{equation}
\begin{split}
    P(T > 30 \text{sec}) &= 1 - P(T \leq 30 \text{sec}) \\
    &= 1 - unifcdf(30, 20, 50) \textit{ (in matlab)} \\
    &= 0.66667 = P2
\end{split}
\end{equation}

Finally, we need to compute the probability that it takes a randomly selected customer more than 30 seconds to connect to the internet. We have that
\begin{equation*}
    \begin{split}
        P(T > 30) &= 80\% P1 + 20\% P2 \\
        &= 80\% \cdot 0.91970 + 20\% \cdot 0.66667 \\
        &= 0.73576 + 0.133334 \\
        &= 0.869094
    \end{split}
\end{equation*}
\end{proof}

\textbf{8.} Let $X, Y \in N(0, 1)$ be independent random variables. Let $D_r$ be the disk centered at the origin with radius r.
Find r such that $P((X, Y ) \in D_r) = 0.3$.

\begin{proof}
The pdf of X is $f_{X}(x) = \frac{e^{-\frac{{x}^{2}}{2}}}{\sqrt{2\pi}}$. 
The pdf of Y is $f_{Y}(y) = \frac{e^{-\frac{{y}^{2}}{2}}}{\sqrt{2\pi}}$.
Since X and Y are independent, $f_{X, Y}(x, y) = f_{X}(x) \cdot f_{Y}(y) = \frac{e^{-\frac{x^2 + y^2}{2}}}{2\pi}$

We know that $P((X, Y) \in D_r) = \int _{D_r} \int f(x, y) \, dx \, dy$.

A disk $D_r$ is defined with the formula \[ x^2 + y^2 \leq r^2 \]
Thus, \[ P(D_r) = 0 \text{, if } x^2 + y^2 > r^2 \]
For $x^2 + y^2 \leq r^2$, we have
\begin{equation*}
\begin{split}
    P(D_r) &= \int_{-r}^{r} \int_{-\sqrt{r^2 - x^2}}^{\sqrt{r^2 - x^2}} \frac{e^{-\frac{x^2 + y^2}{2}}}{2\pi} \, dy \, dx \\
    &= \int_{0}^{r} \int_{0}^{2\pi} \frac{e^\frac{-R^2}{2}}{2\pi} R \, d\theta \, dR \\
    &= \int_{0}^{r} \frac{e^\frac{-R^2}{2}}{2\pi} 2\pi R \, dR \\
    &= 1 - e^{-\frac{r^2}{2}}
\end{split}
\end{equation*}
(by writing to polar coordinates) the probabilty of $(X, Y) \in D_r$. This way, we need to compute \begin{equation*}
    \begin{split}
        & P(R = r) = 0.3 \\
        & 1 - e^{-\frac{r^2}{2}} = 0.3 \\
        & e^{-\frac{r^2}{2}} = 0.7 \\
        & -r^2 = 2 ln(0.7) \\
        & r \approx 0.8446 \\
    \end{split}
\end{equation*}
\end{proof}

\end{document}
